% Options for packages loaded elsewhere
\PassOptionsToPackage{unicode}{hyperref}
\PassOptionsToPackage{hyphens}{url}
%
\documentclass[
]{article}
\usepackage{lmodern}
\usepackage{amssymb,amsmath}
\usepackage{ifxetex,ifluatex}
\ifnum 0\ifxetex 1\fi\ifluatex 1\fi=0 % if pdftex
  \usepackage[T1]{fontenc}
  \usepackage[utf8]{inputenc}
  \usepackage{textcomp} % provide euro and other symbols
\else % if luatex or xetex
  \usepackage{unicode-math}
  \defaultfontfeatures{Scale=MatchLowercase}
  \defaultfontfeatures[\rmfamily]{Ligatures=TeX,Scale=1}
\fi
% Use upquote if available, for straight quotes in verbatim environments
\IfFileExists{upquote.sty}{\usepackage{upquote}}{}
\IfFileExists{microtype.sty}{% use microtype if available
  \usepackage[]{microtype}
  \UseMicrotypeSet[protrusion]{basicmath} % disable protrusion for tt fonts
}{}
\makeatletter
\@ifundefined{KOMAClassName}{% if non-KOMA class
  \IfFileExists{parskip.sty}{%
    \usepackage{parskip}
  }{% else
    \setlength{\parindent}{0pt}
    \setlength{\parskip}{6pt plus 2pt minus 1pt}}
}{% if KOMA class
  \KOMAoptions{parskip=half}}
\makeatother
\usepackage{xcolor}
\IfFileExists{xurl.sty}{\usepackage{xurl}}{} % add URL line breaks if available
\IfFileExists{bookmark.sty}{\usepackage{bookmark}}{\usepackage{hyperref}}
\hypersetup{
  pdftitle={Debugging Practice Modules with Solutions},
  hidelinks,
  pdfcreator={LaTeX via pandoc}}
\urlstyle{same} % disable monospaced font for URLs
\usepackage[margin=1in]{geometry}
\usepackage{color}
\usepackage{fancyvrb}
\newcommand{\VerbBar}{|}
\newcommand{\VERB}{\Verb[commandchars=\\\{\}]}
\DefineVerbatimEnvironment{Highlighting}{Verbatim}{commandchars=\\\{\}}
% Add ',fontsize=\small' for more characters per line
\usepackage{framed}
\definecolor{shadecolor}{RGB}{248,248,248}
\newenvironment{Shaded}{\begin{snugshade}}{\end{snugshade}}
\newcommand{\AlertTok}[1]{\textcolor[rgb]{0.94,0.16,0.16}{#1}}
\newcommand{\AnnotationTok}[1]{\textcolor[rgb]{0.56,0.35,0.01}{\textbf{\textit{#1}}}}
\newcommand{\AttributeTok}[1]{\textcolor[rgb]{0.77,0.63,0.00}{#1}}
\newcommand{\BaseNTok}[1]{\textcolor[rgb]{0.00,0.00,0.81}{#1}}
\newcommand{\BuiltInTok}[1]{#1}
\newcommand{\CharTok}[1]{\textcolor[rgb]{0.31,0.60,0.02}{#1}}
\newcommand{\CommentTok}[1]{\textcolor[rgb]{0.56,0.35,0.01}{\textit{#1}}}
\newcommand{\CommentVarTok}[1]{\textcolor[rgb]{0.56,0.35,0.01}{\textbf{\textit{#1}}}}
\newcommand{\ConstantTok}[1]{\textcolor[rgb]{0.00,0.00,0.00}{#1}}
\newcommand{\ControlFlowTok}[1]{\textcolor[rgb]{0.13,0.29,0.53}{\textbf{#1}}}
\newcommand{\DataTypeTok}[1]{\textcolor[rgb]{0.13,0.29,0.53}{#1}}
\newcommand{\DecValTok}[1]{\textcolor[rgb]{0.00,0.00,0.81}{#1}}
\newcommand{\DocumentationTok}[1]{\textcolor[rgb]{0.56,0.35,0.01}{\textbf{\textit{#1}}}}
\newcommand{\ErrorTok}[1]{\textcolor[rgb]{0.64,0.00,0.00}{\textbf{#1}}}
\newcommand{\ExtensionTok}[1]{#1}
\newcommand{\FloatTok}[1]{\textcolor[rgb]{0.00,0.00,0.81}{#1}}
\newcommand{\FunctionTok}[1]{\textcolor[rgb]{0.00,0.00,0.00}{#1}}
\newcommand{\ImportTok}[1]{#1}
\newcommand{\InformationTok}[1]{\textcolor[rgb]{0.56,0.35,0.01}{\textbf{\textit{#1}}}}
\newcommand{\KeywordTok}[1]{\textcolor[rgb]{0.13,0.29,0.53}{\textbf{#1}}}
\newcommand{\NormalTok}[1]{#1}
\newcommand{\OperatorTok}[1]{\textcolor[rgb]{0.81,0.36,0.00}{\textbf{#1}}}
\newcommand{\OtherTok}[1]{\textcolor[rgb]{0.56,0.35,0.01}{#1}}
\newcommand{\PreprocessorTok}[1]{\textcolor[rgb]{0.56,0.35,0.01}{\textit{#1}}}
\newcommand{\RegionMarkerTok}[1]{#1}
\newcommand{\SpecialCharTok}[1]{\textcolor[rgb]{0.00,0.00,0.00}{#1}}
\newcommand{\SpecialStringTok}[1]{\textcolor[rgb]{0.31,0.60,0.02}{#1}}
\newcommand{\StringTok}[1]{\textcolor[rgb]{0.31,0.60,0.02}{#1}}
\newcommand{\VariableTok}[1]{\textcolor[rgb]{0.00,0.00,0.00}{#1}}
\newcommand{\VerbatimStringTok}[1]{\textcolor[rgb]{0.31,0.60,0.02}{#1}}
\newcommand{\WarningTok}[1]{\textcolor[rgb]{0.56,0.35,0.01}{\textbf{\textit{#1}}}}
\usepackage{graphicx,grffile}
\makeatletter
\def\maxwidth{\ifdim\Gin@nat@width>\linewidth\linewidth\else\Gin@nat@width\fi}
\def\maxheight{\ifdim\Gin@nat@height>\textheight\textheight\else\Gin@nat@height\fi}
\makeatother
% Scale images if necessary, so that they will not overflow the page
% margins by default, and it is still possible to overwrite the defaults
% using explicit options in \includegraphics[width, height, ...]{}
\setkeys{Gin}{width=\maxwidth,height=\maxheight,keepaspectratio}
% Set default figure placement to htbp
\makeatletter
\def\fps@figure{htbp}
\makeatother
\setlength{\emergencystretch}{3em} % prevent overfull lines
\providecommand{\tightlist}{%
  \setlength{\itemsep}{0pt}\setlength{\parskip}{0pt}}
\setcounter{secnumdepth}{-\maxdimen} % remove section numbering

\title{Debugging Practice Modules with Solutions}
\author{}
\date{\vspace{-2.5em}}

\begin{document}
\maketitle

\hypertarget{please-reach-out-on-the-stem-away-forums-with-any-questions.}{%
\paragraph{Please reach out on the STEM-Away forums with any
questions.}\label{please-reach-out-on-the-stem-away-forums-with-any-questions.}}

These modules will hopefully help you get a handle on debugging code.
They are intended as an introduction to debugging to help you get a
sense for the process. If you like, you can download the accompanying R
script, here, and try debugging it without the modules.

The process of debugging is a huge component of programming and it's
important you know how to deal with errors in your code. However, not
all errors will return error messages telling you what's wrong. Let me
introduce the three types of errors you should be aware of:

\begin{enumerate}
\def\labelenumi{\arabic{enumi}.}
\tightlist
\item
  \textbf{Syntax errors}: These errors are like ``spelling and grammar''
  errors in English. They concern the \emph{syntax} of the programming
  language you're using. For instance, forgetting to close a parenthesis
  when using a function:
\end{enumerate}

These are the easiest to handle since they output error messages telling
you what's wrong, and most times, \textbf{where} it's is wrong.

\begin{enumerate}
\def\labelenumi{\arabic{enumi}.}
\setcounter{enumi}{1}
\tightlist
\item
  \textbf{Semantic errors}: These errors have to do with the meaning and
  context. They are often caused by type mismatches. For example, the
  \texttt{sum()} function expects numeric input. If a character or
  string is given, the function will return an error.
\end{enumerate}

\begin{enumerate}
\def\labelenumi{\arabic{enumi}.}
\setcounter{enumi}{2}
\tightlist
\item
  \textbf{Logical errors}: These errors have to do with the program
  flow. A classic example is found in learning PEMDAS, the order of
  operations in math equations.
\end{enumerate}

These errors are often the hardest to identify and debug because they
typically do not return an error message. Logical errors are identified
when a program or function has an unintended output.

\hypertarget{debugging-tips}{%
\section{Debugging Tips}\label{debugging-tips}}

\begin{itemize}
\tightlist
\item
  Check spellings.
\item
  Ensure that your variables contain the information you expect.
\item
  Check that all brackets are closed.
\item
  Ensure that all data and libraries are loaded correctly.
\item
  If you keep trying and trying and nothing seems to work, try clearing
  your environment (there's a little broom icon at the top of the
  Environment tab) and running your working code again.
\end{itemize}

\hypertarget{debugging-modules}{%
\section{Debugging Modules}\label{debugging-modules}}

\hypertarget{module-1-lines-17-18}{%
\subsubsection{Module 1 (Lines 17-18)}\label{module-1-lines-17-18}}

The cat\_data data frame should look something like this:

\hypertarget{hints---click-on-the-code-buttons-to-the-right-to-show-hints.}{%
\paragraph{Hints - click on the code buttons to the right to show
hints.}\label{hints---click-on-the-code-buttons-to-the-right-to-show-hints.}}

\begin{Shaded}
\begin{Highlighting}[]
\CommentTok{# What is the name of the csv you are trying to access? Check your directory.}
\end{Highlighting}
\end{Shaded}

\begin{Shaded}
\begin{Highlighting}[]
\CommentTok{# What does `header=F` do? Check the documentation for `read.csv()`.}
\end{Highlighting}
\end{Shaded}

\begin{Shaded}
\begin{Highlighting}[]
\CommentTok{# What does `row.names=NULL` do? Check the documentation for `read.csv()`.}
\end{Highlighting}
\end{Shaded}

\hypertarget{solution---click-on-the-code-button-to-the-right-to-show-a-solution}{%
\paragraph{Solution - click on the code button to the right to show a
solution}\label{solution---click-on-the-code-button-to-the-right-to-show-a-solution}}

\begin{Shaded}
\begin{Highlighting}[]
\NormalTok{cat_data <-}\StringTok{ }\NormalTok{cats }\CommentTok{#or}
\NormalTok{cat_data <-}\StringTok{ }\KeywordTok{read.csv}\NormalTok{(}\StringTok{"cats.csv"}\NormalTok{, }\DataTypeTok{header=}\NormalTok{T, }\DataTypeTok{row.names=}\DecValTok{1}\NormalTok{)}
\end{Highlighting}
\end{Shaded}

\hypertarget{module-2-lines-21-23}{%
\subsubsection{Module 2 (Lines 21-23)}\label{module-2-lines-21-23}}

The expected output should look something like this:

{[}1{]} ``M''

{[}1{]} 3.9

Bwt Hwt 144 3.9 20.5

\hypertarget{hints---click-on-the-code-buttons-to-the-right-to-show-hints.-1}{%
\paragraph{Hints - click on the code buttons to the right to show
hints.}\label{hints---click-on-the-code-buttons-to-the-right-to-show-hints.-1}}

\begin{Shaded}
\begin{Highlighting}[]
\CommentTok{# How do you index a vector?}
\end{Highlighting}
\end{Shaded}

\begin{Shaded}
\begin{Highlighting}[]
\CommentTok{# How do you do negative indexing?}
\end{Highlighting}
\end{Shaded}

\hypertarget{solution---click-on-the-code-button-to-the-right-to-show-a-solution.}{%
\paragraph{Solution - click on the code button to the right to show a
solution.}\label{solution---click-on-the-code-button-to-the-right-to-show-a-solution.}}

\begin{Shaded}
\begin{Highlighting}[]
\NormalTok{cat_data}\OperatorTok{$}\NormalTok{Sex[}\DecValTok{144}\NormalTok{]}
\NormalTok{cat_data}\OperatorTok{$}\NormalTok{Bwt[}\OperatorTok{-}\KeywordTok{c}\NormalTok{(}\DecValTok{1}\OperatorTok{:}\DecValTok{143}\NormalTok{)] }\CommentTok{#negative indexing}
\NormalTok{cat_data[}\DecValTok{144}\NormalTok{, }\DecValTok{2}\OperatorTok{:}\DecValTok{3}\NormalTok{]}
\end{Highlighting}
\end{Shaded}

\hypertarget{module-3-lines-27-30}{%
\subsubsection{Module 3 (Lines 27-30)}\label{module-3-lines-27-30}}

\hypertarget{hint---click-on-the-code-button-to-the-right-to-show-a-hint.}{%
\paragraph{Hint - click on the code button to the right to show a
hint.}\label{hint---click-on-the-code-button-to-the-right-to-show-a-hint.}}

\begin{Shaded}
\begin{Highlighting}[]
\CommentTok{# How do you index a data frame?}
\end{Highlighting}
\end{Shaded}

\hypertarget{solution---click-on-the-code-button-to-the-right-to-show-a-solution.-1}{%
\paragraph{Solution - click on the code button to the right to show a
solution.}\label{solution---click-on-the-code-button-to-the-right-to-show-a-solution.-1}}

\begin{Shaded}
\begin{Highlighting}[]
\NormalTok{large_hearts <-}\StringTok{ }\NormalTok{cat_data[cat_data}\OperatorTok{$}\NormalTok{Hwt }\OperatorTok{>}\StringTok{ }\DecValTok{12}\NormalTok{, ] }\CommentTok{#or}
\NormalTok{large_hearts <-}\StringTok{ }\KeywordTok{subset}\NormalTok{(cat_data, Hwt }\OperatorTok{>}\StringTok{ }\DecValTok{12}\NormalTok{)}

\NormalTok{small_hearts <-}\StringTok{ }\KeywordTok{subset}\NormalTok{(cat_data, Hwt }\OperatorTok{<}\StringTok{ }\DecValTok{8}\NormalTok{)}
\end{Highlighting}
\end{Shaded}

\hypertarget{module-4-lines-33-38}{%
\subsubsection{Module 4 (Lines 33-38)}\label{module-4-lines-33-38}}

The cat\_data data frame should now look something like this:

\hypertarget{hints---click-on-the-code-buttons-to-the-right-to-show-hints.-2}{%
\paragraph{Hints - click on the code buttons to the right to show
hints.}\label{hints---click-on-the-code-buttons-to-the-right-to-show-hints.-2}}

\begin{Shaded}
\begin{Highlighting}[]
\CommentTok{# What is the length of `hwt_class`? What is the length of a column in `cat_data`. Remember that data frames are a collection of vectors of the same length.}
\end{Highlighting}
\end{Shaded}

\begin{Shaded}
\begin{Highlighting}[]
\CommentTok{# Check spelling.}
\end{Highlighting}
\end{Shaded}

\hypertarget{solution---click-on-the-code-button-to-the-right-to-show-a-solution-1}{%
\paragraph{Solution - click on the code button to the right to show a
solution}\label{solution---click-on-the-code-button-to-the-right-to-show-a-solution-1}}

\begin{Shaded}
\begin{Highlighting}[]
\NormalTok{hwt_class <-}\StringTok{ }\KeywordTok{rep}\NormalTok{(}\StringTok{"M"}\NormalTok{, }\DecValTok{144}\NormalTok{)}
\NormalTok{large_ids <-}\StringTok{ }\KeywordTok{which}\NormalTok{(cat_data}\OperatorTok{$}\NormalTok{Hwt }\OperatorTok{>}\StringTok{ }\DecValTok{12}\NormalTok{)}
\NormalTok{hwt_class[large_ids] <-}\StringTok{ "L"}
\NormalTok{small_ids <-}\StringTok{ }\KeywordTok{which}\NormalTok{(cat_data}\OperatorTok{$}\NormalTok{Hwt }\OperatorTok{<}\StringTok{ }\DecValTok{8}\NormalTok{)}
\NormalTok{hwt_class[small_ids] <-}\StringTok{ "S"}
\NormalTok{cat_data}\OperatorTok{$}\NormalTok{Hwt_class <-}\StringTok{ }\NormalTok{hwt_class}
\end{Highlighting}
\end{Shaded}

\hypertarget{module-5-line-41}{%
\subsubsection{Module 5 (Line 41)}\label{module-5-line-41}}

\hypertarget{hint---click-on-the-code-button-to-the-right-to-show-a-hint.-1}{%
\paragraph{Hint - click on the code button to the right to show a
hint.}\label{hint---click-on-the-code-button-to-the-right-to-show-a-hint.-1}}

\begin{Shaded}
\begin{Highlighting}[]
\CommentTok{# Check spelling.}
\end{Highlighting}
\end{Shaded}

\hypertarget{solution---click-on-the-code-button-to-the-right-to-show-a-solution.-2}{%
\paragraph{Solution - click on the code button to the right to show a
solution.}\label{solution---click-on-the-code-button-to-the-right-to-show-a-solution.-2}}

\begin{Shaded}
\begin{Highlighting}[]
\KeywordTok{write.csv}\NormalTok{(cat_data, }\StringTok{"./cats_edited.csv"}\NormalTok{)}
\end{Highlighting}
\end{Shaded}

\hypertarget{module-6-lines-45-52}{%
\subsubsection{Module 6 (Lines 45-52)}\label{module-6-lines-45-52}}

The code above should produce a plot that looks like this:

\hypertarget{hints---click-on-the-code-buttons-to-the-right-to-show-hints.-3}{%
\paragraph{Hints - click on the code buttons to the right to show
hints.}\label{hints---click-on-the-code-buttons-to-the-right-to-show-hints.-3}}

\begin{Shaded}
\begin{Highlighting}[]
\CommentTok{# Do the aesthetics in the `ggplot()` function correspond to those in the `scale_*fill*_brewer()` function?}
\end{Highlighting}
\end{Shaded}

\begin{Shaded}
\begin{Highlighting}[]
\CommentTok{# Are there alternative settings for position in the geom_bar() function?}
\end{Highlighting}
\end{Shaded}

\begin{Shaded}
\begin{Highlighting}[]
\CommentTok{# If you store a plot as an object, there's one more step you need to do for writing it to a file. If you're stuck, check out the Graphs section of Cookbook for R.}
\end{Highlighting}
\end{Shaded}

\hypertarget{solution---click-on-the-button-to-the-right-to-show-a-solution.}{%
\paragraph{Solution - click on the button to the right to show a
solution.}\label{solution---click-on-the-button-to-the-right-to-show-a-solution.}}

\begin{Shaded}
\begin{Highlighting}[]
\KeywordTok{png}\NormalTok{(}\StringTok{"frequency_plot.png"}\NormalTok{, }\DataTypeTok{width=}\DecValTok{1000}\NormalTok{, }\DataTypeTok{height=}\DecValTok{1000}\NormalTok{)}
\NormalTok{freq <-}\StringTok{ }\KeywordTok{ggplot}\NormalTok{(}\DataTypeTok{data=}\NormalTok{cat_data, }\KeywordTok{aes}\NormalTok{(}\DataTypeTok{x=}\NormalTok{Hwt_class, }\DataTypeTok{fill=}\NormalTok{Sex))}\OperatorTok{+}
\StringTok{  }\KeywordTok{geom_bar}\NormalTok{(}\DataTypeTok{stat=}\StringTok{"count"}\NormalTok{, }\DataTypeTok{position=}\KeywordTok{position_dodge}\NormalTok{())}\OperatorTok{+}
\StringTok{  }\KeywordTok{xlab}\NormalTok{(}\StringTok{"Heart Size"}\NormalTok{)}\OperatorTok{+}\StringTok{ }\KeywordTok{ylab}\NormalTok{(}\StringTok{"Count"}\NormalTok{)}\OperatorTok{+}
\StringTok{  }\KeywordTok{ggtitle}\NormalTok{(}\StringTok{"Number of Cats in Each Size Classification"}\NormalTok{)}\OperatorTok{+}
\StringTok{  }\KeywordTok{scale_fill_manual}\NormalTok{(}\DataTypeTok{values=}\KeywordTok{c}\NormalTok{(}\StringTok{"midnightblue"}\NormalTok{, }\StringTok{"mediumturquoise"}\NormalTok{))}\OperatorTok{+}
\StringTok{  }\KeywordTok{theme_bw}\NormalTok{()}
\KeywordTok{print}\NormalTok{(freq)}
\KeywordTok{dev.off}\NormalTok{()}
\end{Highlighting}
\end{Shaded}

\hypertarget{module-7-lines-55-61}{%
\subsubsection{Module 7 (Lines 55-61)}\label{module-7-lines-55-61}}

The Hwt data frame should look something like this:

\hypertarget{hints---click-on-the-code-buttons-to-the-right-to-show-hints.-4}{%
\paragraph{Hints - click on the code buttons to the right to show
hints.}\label{hints---click-on-the-code-buttons-to-the-right-to-show-hints.-4}}

\begin{Shaded}
\begin{Highlighting}[]
\CommentTok{# What kind of object is returned by `cat_data[cat_data$Hwt_class == "S",]`? }
\end{Highlighting}
\end{Shaded}

\begin{Shaded}
\begin{Highlighting}[]
\CommentTok{# What kind of argument does `sd()` expect?}
\end{Highlighting}
\end{Shaded}

\begin{Shaded}
\begin{Highlighting}[]
\CommentTok{# How many elements are in `avg`? How many elements are in `se`? Remember that data frames are a collection of vectors of the same length.}
\end{Highlighting}
\end{Shaded}

\begin{Shaded}
\begin{Highlighting}[]
\CommentTok{# R is case sensitive. Are there any instances where the code is trying to refer to an object but mismatches the case?}
\end{Highlighting}
\end{Shaded}

\hypertarget{solution---click-on-the-code-button-to-the-right-to-show-a-solution-2}{%
\paragraph{Solution - click on the code button to the right to show a
solution}\label{solution---click-on-the-code-button-to-the-right-to-show-a-solution-2}}

\begin{Shaded}
\begin{Highlighting}[]
\NormalTok{avg <-}\StringTok{ }\KeywordTok{c}\NormalTok{(}\KeywordTok{mean}\NormalTok{(cat_data[cat_data}\OperatorTok{$}\NormalTok{Hwt_class }\OperatorTok{==}\StringTok{ "S"}\NormalTok{,}\DecValTok{2}\NormalTok{]),}
         \KeywordTok{mean}\NormalTok{(cat_data[cat_data}\OperatorTok{$}\NormalTok{Hwt_class }\OperatorTok{==}\StringTok{ "M"}\NormalTok{,}\DecValTok{2}\NormalTok{]),}
         \KeywordTok{mean}\NormalTok{(cat_data[cat_data}\OperatorTok{$}\NormalTok{Hwt_class }\OperatorTok{==}\StringTok{ "L"}\NormalTok{,}\DecValTok{2}\NormalTok{]))}
\NormalTok{se <-}\StringTok{ }\KeywordTok{c}\NormalTok{(}\KeywordTok{sd}\NormalTok{(cat_data[cat_data}\OperatorTok{$}\NormalTok{Hwt_class }\OperatorTok{==}\StringTok{ "S"}\NormalTok{,}\DecValTok{2}\NormalTok{]),}
        \KeywordTok{sd}\NormalTok{(cat_data[cat_data}\OperatorTok{$}\NormalTok{Hwt_class }\OperatorTok{==}\StringTok{ "M"}\NormalTok{,}\DecValTok{2}\NormalTok{]),}
        \KeywordTok{sd}\NormalTok{(cat_data[cat_data}\OperatorTok{$}\NormalTok{Hwt_class }\OperatorTok{==}\StringTok{ "L"}\NormalTok{,}\DecValTok{2}\NormalTok{]))}
\NormalTok{Hwt <-}\StringTok{ }\KeywordTok{data.frame}\NormalTok{(}\DataTypeTok{Class=}\KeywordTok{c}\NormalTok{(}\StringTok{"S"}\NormalTok{, }\StringTok{"M"}\NormalTok{, }\StringTok{"L"}\NormalTok{),}
                      \DataTypeTok{Avg=}\NormalTok{avg, }\DataTypeTok{SE=}\NormalTok{se)}
\end{Highlighting}
\end{Shaded}

\hypertarget{module-8-lines-63-72}{%
\subsubsection{Module 8 (Lines 63-72)}\label{module-8-lines-63-72}}

The code above should produce a plot that looks like this:

\hypertarget{hints---click-on-the-code-buttons-to-the-right-to-show-hints.-5}{%
\paragraph{Hints - click on the code buttons to the right to show
hints.}\label{hints---click-on-the-code-buttons-to-the-right-to-show-hints.-5}}

\begin{Shaded}
\begin{Highlighting}[]
\CommentTok{# Are there alternative settings for stat in the `geom_bar()` function?}
\end{Highlighting}
\end{Shaded}

\begin{Shaded}
\begin{Highlighting}[]
\CommentTok{# What does width in the `geom_errorbar()` function do?}
\end{Highlighting}
\end{Shaded}

\begin{Shaded}
\begin{Highlighting}[]
\CommentTok{# Where are the modified labels and title? Pay attention to what follows the `guides()` function.}
\end{Highlighting}
\end{Shaded}

\begin{Shaded}
\begin{Highlighting}[]
\CommentTok{# Look at the documentation for `scale_fill_brewer()`. What is `"Accent"` referring to?}
\end{Highlighting}
\end{Shaded}

\hypertarget{solution---click-on-the-code-button-to-the-right-to-show-a-solution.-3}{%
\paragraph{Solution - click on the code button to the right to show a
solution.}\label{solution---click-on-the-code-button-to-the-right-to-show-a-solution.-3}}

\begin{Shaded}
\begin{Highlighting}[]
\KeywordTok{ggplot}\NormalTok{(}\DataTypeTok{data=}\NormalTok{Hwt, }\KeywordTok{aes}\NormalTok{(}\DataTypeTok{x=}\NormalTok{Class, }\DataTypeTok{y=}\NormalTok{Avg, }\DataTypeTok{fill=}\NormalTok{Class))}\OperatorTok{+}
\StringTok{  }\KeywordTok{geom_bar}\NormalTok{(}\DataTypeTok{stat=}\StringTok{"identity"}\NormalTok{)}\OperatorTok{+}
\StringTok{  }\KeywordTok{geom_errorbar}\NormalTok{(}\KeywordTok{aes}\NormalTok{(}\DataTypeTok{ymin=}\NormalTok{Avg}\OperatorTok{-}\NormalTok{SE, }\DataTypeTok{ymax=}\NormalTok{Avg}\OperatorTok{+}\NormalTok{SE), }
                \DataTypeTok{width=}\NormalTok{.}\DecValTok{2}\NormalTok{)}\OperatorTok{+}
\StringTok{  }\KeywordTok{guides}\NormalTok{(}\DataTypeTok{fill=}\NormalTok{F)}\OperatorTok{+}
\StringTok{  }\KeywordTok{xlab}\NormalTok{(}\StringTok{"Heart Size"}\NormalTok{)}\OperatorTok{+}\StringTok{ }\KeywordTok{ylab}\NormalTok{(}\StringTok{"Average Body Weight (kg)"}\NormalTok{)}\OperatorTok{+}
\StringTok{  }\KeywordTok{ggtitle}\NormalTok{(}\StringTok{"Average Body Weights by Size Calssifaction"}\NormalTok{)}\OperatorTok{+}
\StringTok{  }\KeywordTok{scale_fill_brewer}\NormalTok{(}\DataTypeTok{palette=}\StringTok{"Accent"}\NormalTok{)}\OperatorTok{+}
\StringTok{  }\KeywordTok{theme_bw}\NormalTok{()}
\KeywordTok{ggsave}\NormalTok{(}\StringTok{"avg_plot.png"}\NormalTok{, }\DataTypeTok{width=}\DecValTok{5}\NormalTok{, }\DataTypeTok{height=}\DecValTok{5}\NormalTok{)}
\end{Highlighting}
\end{Shaded}

\end{document}
